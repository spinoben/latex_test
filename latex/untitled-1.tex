\documentclass[11pt, oneside, a4paper]{article}
\usepackage{makeidx}
\makeindex
\pagestyle{headings}
\author{qwer \and tyui}
\title{1122}
\newtheorem{law}{Law}[section]
\newtheorem{jury}{Jury}
\begin{document}
\maketitle
\newpage
\tableofcontents
\section{sec1}
hello, world
\subsection{sec1.1}
\section{sec2}
\ldots{} when 
\begin{equation}
    e = m\cdot c^2 \; 
\end{equation}
\ldots{}end\\
\fbox{\today I think it is supercalifragilisticexpialidocious}
% $...$ turn to math mode in text, $$..$$ will turn to math mode in new line
\\ ``\,`asdaslfgakasda'\,'' \\ 0 - 2 = 4 \\ 13--6 \\ 13---6 \\ \~ 13$-6$ \\ is: $$13-6$$\\
% \mathrm generate the text style, when in math mode, all will be math style
$-30\,^{\circ}\mathrm{C}$ .c \\
% textit(italic), textup(upright), textsl(slant), textbf(bold)
\underline{the the} \emph{that that} \textit{text \emph{text}} \\
\footnote{this}
\newpage
%below is the environment of \begin{...} until end
% list:itemize(with dot), enumerate(with num), description(simple table)
\begin{itemize}
    \item qwerty
    % item[...] can change the index of dot or num
    \item[one] sadsag
\end{itemize} 
\begin{enumerate}
    \item qwerty
    \item sadsag
\end{enumerate} 
\begin{description}
    \item qwerty
    \item sadsag
\end{description} 
% flushleft(align left), flushright(align right), center(algin center)
\begin{flushleft}
    asdf \\ fgh
\end{flushleft}
\begin{center}
    asdf \\ fgh
\end{center}
\begin{flushright}
    asdf \\ fgh
\end{flushright}
% table: tabular
\begin{tabular}{|l|c|r|}    %'|' means upright line, 'l,c,r' means this column align in left, center, right
    \hline    %horizontal line
    1 & 2 & 3\\    % '&' means jump to another column
    \hline \hline
    4 & 5 & 6\\    %double horizontal line, will divide in two table
    \hline
\end{tabular}

\begin{tabular}{c}    %'|' means upright line, 'l,c,r' means this column align in left, center, right
    \hline    %horizontal line
    1 3\\    % '&' means jump to another column
    \hline
\end{tabular}

% environment end

%math
%similar to $$
\begin{math}
    c^{2} = b^{2} + a^{2}
\end{math}
qwertyu
%displatmath will center and take up one line , but without label unless using equation
\begin{displaymath}
    c^{2} = b^{2} + a^{2}
\end{displaymath}
%using \label or \ref to quote
\begin{equation}\label{eps}
    \epsilon > 0
\end{equation} 
From \ref{eps}, we can get \ldots{}
% some difference with $$
$\lim_{n \to \infty} \sum_{k=1}^n \frac{1}{k^2} = \frac{\pi^2}{6} $
\begin{displaymath}
    % if want to display text in displaymath, using \textrm{text}, using \, \quad, \qquad
    \lim_{n \to \infty} \sum_{k=1}^n \frac{1}{k^2} = \frac{\pi^2}{6} \textrm{in all}
\end{displaymath}
% greek letters    \alpha \Delta \beta  \gamma \Gamma ...
$ \alpha \Delta $
$ a_{1} b^{2} \sqrt[3]{x} \quad a\cdot b \quad \overline{m+n} \underline{m+n} \overbrace{a+b+...}^{26} \underbrace{a+b+...}_{26} $
% \vec or \overrightarrow is used to define vector, but \vec only function in one variable
$ \vec abc \quad \overrightarrow{a+b+c} \\ $
% {... \choose ...}, {... \atop ...}, \stackrel
$ {n \choose k}\qquad {x \atop y+2} \quad \int a+b\stackrel{!}{=}c $
% integral operator    \int_{}^{}, \sum, \prod
$ \int_{-2}^{5} $
% \left( ... \right) will adjust the size of brace. If only single '(', \right .
\begin{displaymath}
    1 + \left( \frac{1}{ 1-x^{2} }\right) ^3 \quad
    1 + ( \frac{1}{ 1-x^{2} }) ^3
\end{displaymath}
% display matrix
\begin{displaymath}
    \mathbf{X} = 
    \left ( 
        \begin{array}{ccc} % ccc can replace by c|c|c(will show upright line)
            x_{11} & x_{12} & \ldots{} \\  % can add \hline
            x_{21} & x_{22} & \ldots{} \\
            \vdots & \vdots & \ddots
        \end{array} 
    \right)
\end{displaymath}
% display piecewise function
\begin{displaymath}
    \mathbf{y} = 
    \left \{
        \begin{array}{ccc}
            1 & \textrm{if $x>b$} \\
            2 & \textrm{if $x<b$}
        \end{array}
    \right .
\end{displaymath}
% eqnarray will generate equation system
\begin{eqnarray}
    1 + 1 =2\\1 + 2 = 3 \\ 1 + 3 = 4
\end{eqnarray}
% phantom{1} will generate '1' but will not display
\begin{displaymath}
    {}^{12}_{6} \textrm{C} \quad {}^{12}_{\phantom{1}6} \textrm{C}
\end{displaymath}
% define theorem: \newtheorem{law}{123456}[section]
\begin{law}
    QWERTY
\end{law}
\begin{jury}
    ASDFG
\end{jury}
% mathbf, boldmath: two kinds of bold. mathbf use the math type, boldmath only add bold and only use in text
\begin{displaymath}
    \mu, M \qquad \mathbf{\mu, M} 
\end{displaymath}
\boldmath {$\mu, M$}
% bibliography
Part1 \cite{1} has proposed that \ldots
\begin{thebibliography}{86}
    \bibitem{1} H.~P
\end{thebibliography}

\end{document}